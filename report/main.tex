\documentclass[a4paper, titlepage]{article}
\usepackage[utf8]{inputenc}
\usepackage[italian]{babel}
%\usepackage[hashEnumerators,smartEllipses]{markdown}
\usepackage{mathtools}
\usepackage{physics}    
%\usepackage{amsmath} %mathtools loads amsmath too!!
\usepackage{amssymb}
\usepackage{listings}
\usepackage{tabularx}
\usepackage{textcomp}
\usepackage{siunitx}
\usepackage{multirow}
\usepackage{multicol}
\usepackage{booktabs}
\usepackage{graphicx}
\usepackage{floatflt}
\usepackage{epsfig}
\usepackage{pstricks}
\usepackage{subcaption}
\usepackage[labelfont=bf, font=scriptsize]{caption}
\usepackage[italian]{varioref}
\usepackage[suftesi,write]{frontespizio}
\usepackage{color}
\usepackage{caption}
\usepackage{pgfplots}
\usepackage{comment}
\usepackage{bm}            % special bold-math package. usge: \bm{mathsymbol}
\usepackage{array}
\usepackage{lipsum}
\usepackage{csquotes}
\usepackage{sidecap}
\usepackage{biblatex}
\usepackage[version=4]{mhchem}
%\addbibresource{sample-paper.bib}
\usepackage[colorlinks=true]{hyperref}  % this package should be added after all others.
\pgfplotsset{compat=1.16}
\usepackage[text={15.5cm,23.5cm},centering,heightrounded]{geometry}
\DeclareCaptionType{eq_caption}[Equazione][Elenco delle equazioni]

%font
%\usepackage{tgpagella}
\usepackage[T1]{fontenc}
%\usepackage{lmodern}

\definecolor{mygreen}{rgb}{0,0.6,0}
\definecolor{mygray}{rgb}{0.5,0.5,0.5}
\definecolor{mymauve}{rgb}{0.58,0,0.82}

\lstset{ 
  backgroundcolor=\color{white},   % choose the background color; you must add \usepackage{color} or \usepackage{xcolor}; should come as last argument
  basicstyle=\footnotesize,        % the size of the fonts that are used for the code
  breakatwhitespace=false,         % sets if automatic breaks should only happen at whitespace
  breaklines=true,                 % sets automatic line breaking
  captionpos=b,                    % sets the caption-position to bottom
  commentstyle=\color{mygreen},    % comment style
  deletekeywords={...},            % if you want to delete keywords from the given language
  escapeinside={\%*}{*)},          % if you want to add LaTeX within your code
  extendedchars=true,              % lets you use non-ASCII characters; for 8-bits encodings only, does not work with UTF-8
  firstnumber=1,                   % start line enumeration with line 1000
  frame=single,	                 % adds a frame around the code
  keepspaces=true,                 % keeps spaces in text, useful for keeping indentation of code (possibly needs columns=flexible)
  keywordstyle=\color{blue},       % keyword style
  %language=Octave,                % the language of the code
  morekeywords={*,...},            % if you want to add more keywords to the set
  numbers=left,                    % where to put the line-numbers; possible values are (none, left, right)
  numbersep=5pt,                   % how far the line-numbers are from the code
  numberstyle=\tiny\color{mygray}, % the style that is used for the line-numbers
  rulecolor=\color{black},         % if not set, the frame-color may be changed on line-breaks within not-black text (e.g. comments (green here))
  showspaces=false,                % show spaces everywhere adding particular underscores; it overrides 'showstringspaces'
  showstringspaces=false,          % underline spaces within strings only
  showtabs=false,                  % show tabs within strings adding particular underscores
  stepnumber=1,                    % the step between two line-numbers. If it's 1, each line will be numbered
  stringstyle=\color{mymauve},     % string literal style
  tabsize=2,	                   % sets default tabsize to 2 spaces
  title=\lstname                   % show the filename of files included with \lstinputlisting; also try caption instead of title
}

%%% Il documento vero e proprio %%%
\begin{document}
\begin{frontespizio}
\Universita{Trento} % CTT
\Logo{Figures/logo_unitn} % CTT
\Divisione{Fisica Computazionale} % CTT
\Corso[Laurea Triennale]{Fisica} % CTT, a meno che non cambi la denominazione del corso
\Annoaccademico{2023-2024}
\Titoletto{Relazione di laboratorio} % CTT
\Titolo{Progetto finale:\\ Monte Carlo Variazionale per gocce di elio}
\Sottotitolo{\today}
\NCandidati{} 
%\Candidato[227552]{Federico De Paoli, \textsf {federico.depaoli@studenti.unitn.it}}
%\Candidato[230338]{Anna Do', \textsf {anna.do@studenti.unitn.it}}
\Candidato[227051]{Giorgio Micaglio, \textsf {giorgio.micaglio@studenti.unitn.it}}
\NRelatore{Docente}{} % CTT
\Relatore{Prof. Alessandro Roggero} % CTT, a meno che non sia cambiato il Prof.
\end{frontespizio}
\IfFileExists{\jobname-frn.pdf}{}{%
\immediate\write18{pdflatex \jobname-frn}} % ASSOLUTAMENTE CTT, è il comando che materialmente vi genera il frontespizio.

\newpage
\newcommand{\sch}[0]{Schrödinger }
%\newcommand{\vv}[0]{Velocity-Verlet }
\newcommand{\lj}[0]{Lennard-Jones }
\newcommand{\boldalpha}[0]{{\boldsymbol{\alpha}}}

\section{Introduzione}
L'obiettivo di questo progetto è calcolare l'energia dello stato fondamentale di un sistema di $^4\ce{He}$ in un potenziale esterno armonico con il metodo di Monte Carlo Variazionale. L'operatore hamiltoniano che descrive il sistema è dunque
\begin{equation*}
    H = -\frac{\hbar^2}{2m} \sum_{i = 1}^N \nabla_i^2 + \frac{1}{2} m\omega^2 \sum_{i = 1}^N r_i^2 + \sum_{i < j} V(r_{ij}) 
    \quad\text{con}\quad 
    V(r) = 4\varepsilon\left[\left(\frac{\sigma}{r}\right)^{12} - \left(\frac{\sigma}{r}\right)^{6}\right]\, ,
\end{equation*}
che è il potenziale di \lj classico. Per usare unità di \unit{\angstrom} per le lunghezze e \unit{\kelvin} per le energie, nel caso dell'elio si hanno
\[
\varepsilon = 10.22\ \unit{\kelvin}, \quad
\sigma = 2.556\ \unit{\angstrom}, \quad 
\frac{\hbar^2}{2m} = 6.0596\ \unit{\square\angstrom\kelvin}\, .
\]
Sia $\Psi_{\boldsymbol{\alpha}}(\mathbf{R}) = \langle\mathbf{R}|\Psi_{\boldsymbol{\alpha}}\rangle$ una funzione d'onda parametrica per il sistema, in cui $\mathbf{R} = (\mathbf{r}_1, \mathbf{r}_2, \dots, \mathbf{r}_N)$ e $\boldsymbol{\alpha}$ è il set di parametri liberi. Il metodo variazionale permette di affermare che
\[
\min_{\boldsymbol{\alpha}} E_{\boldsymbol{\alpha}} = \min_{\boldsymbol{\alpha}} \frac{\langle\Psi_{\boldsymbol{\alpha}}|H|\Psi_{\boldsymbol{\alpha}}\rangle}{\langle\Psi_{\boldsymbol{\alpha}}|\Psi_{\boldsymbol{\alpha}}\rangle} \geq E_0
\]
ed $E_{\boldsymbol{\alpha}}$ si può valutare con una simulazione Monte Carlo usando $P(\mathbf{R}) = |\Psi_{\boldsymbol{\alpha}}(\mathbf{R})|^2$, mentre gli osservabili si calcoleranno con
\[
O_\boldalpha = \frac{1}{M}\sum_{k = 1}^M \frac{O\Psi_\boldalpha(\mathbf{R}_k)}{\Psi_\boldalpha(\mathbf{R}_k)}\, .
\]
La scelta intrapresa per la funzione d'onda, con $\boldalpha = (\alpha, \beta_1, \beta_2)$, è 
\[
\Psi_\boldalpha(\mathbf{R}) = \exp\left(-\frac{1}{2\alpha}\sum_{i = 1}^Nr_i^2-\frac{1}{2}\sum_{i < j}u_\beta(r_{ij})\right) \quad\text{con}\quad u_\beta(r) = \left(\frac{\beta_1}{r}\right)^{\beta_2}.
\]

\subsection{Energia cinetica}
Per la forma della funzione d'onda usata, conviene calcolare il contributo della particella i-esima all'energia cinetica nel seguente modo:
\[
T_i = -\frac{\hbar^2}{2m}\left(\nabla_i^2\log\Psi + (\boldsymbol{\nabla}_i\log\Psi)^2\right)\, ,
\]
che in funzione di $u_\beta(r_{ij})$, dove $r_{ij} = |\mathbf{r}_i - \mathbf{r}_j|$, e delle sue derivate prima e seconda diventa
\[
T_i = \frac{\hbar^2}{2m}
\left[
\frac{3}{\alpha} + 
\frac{1}{2}\sum_{j\neq i} u_\beta''(r_{ij}) + 
\sum_{j\neq i} \frac{u_\beta'(r_{ij})}{r_{ij}} -
\frac{1}{\alpha^2}r_i^2 - 
\frac{1}{\alpha} \sum_{j\neq i} u_\beta'(r_{ij}) \mathbf{r}_i \cdot \hat{\mathbf{r}}_{ij} - 
\frac{1}{4}\left(\sum_{j\neq i} u_\beta'(r_{ij}) \hat{\mathbf{r}}_{ij}\right)^2
\right]\, .
\]

\subsection{Energia totale per 2 particelle}
Nel caso in cui $N = 2$, i contributi all'energia cinetica sono esprimibili in modo semplice, e l'energia totale risulta
\begin{equation}
    \begin{split}
    H = &\frac{\hbar^2}{2m}
    \left[
    \frac{6}{\alpha} +
    u_\beta''(r_{12}) +
    2\frac{u_\beta'(r_{12})}{r_{12}} -
    \frac{1}{\alpha^2}(r_1^2 + r_2^2) - 
    \frac{1}{\alpha}r_{12}u_\beta'(r_{12}) - 
    \frac{1}{2}(u_\beta'(r_{12}))^2
    \right] + \\
    &+\frac{1}{2}m\omega^2(r_1^2 + r_2^2) + 
    4\varepsilon\left[\left(\frac{\sigma}{r_{12}}\right)^{12} - \left(\frac{\sigma}{r_{12}}\right)^{6}\right].
    \end{split}
\end{equation}
Inoltre, si vuole che l'energia sia finita per $r_{12}\rightarrow 0$, e da questo si può fissare il parametro $\beta_2$

\end{document}