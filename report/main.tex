\documentclass[a4paper, titlepage]{article}
\input{packages}

%%% Il documento vero e proprio %%%
\begin{document}
\begin{frontespizio}
\Universita{Trento} % CTT
\Logo{Figures/logo_unitn} % CTT
\Divisione{Fisica Computazionale} % CTT
\Corso[Laurea Triennale]{Fisica} % CTT, a meno che non cambi la denominazione del corso
\Annoaccademico{2023-2024}
\Titoletto{Relazione di laboratorio} % CTT
\Titolo{Progetto finale:\\ Monte Carlo Variazionale per gocce di elio}
\Sottotitolo{\today}
\NCandidati{} 
%\Candidato[227552]{Federico De Paoli, \textsf {federico.depaoli@studenti.unitn.it}}
%\Candidato[230338]{Anna Do', \textsf {anna.do@studenti.unitn.it}}
\Candidato[227051]{Giorgio Micaglio, \textsf {giorgio.micaglio@studenti.unitn.it}}
\NRelatore{Docente}{} % CTT
\Relatore{Prof. Alessandro Roggero} % CTT, a meno che non sia cambiato il Prof.
\end{frontespizio}
\IfFileExists{\jobname-frn.pdf}{}{%
\immediate\write18{pdflatex \jobname-frn}} % ASSOLUTAMENTE CTT, è il comando che materialmente vi genera il frontespizio.

\newpage
\newcommand{\sch}[0]{Schrödinger }
%\newcommand{\vv}[0]{Velocity-Verlet }
\newcommand{\lj}[0]{Lennard-Jones }
\newcommand{\boldalpha}[0]{{\boldsymbol{\alpha}}}

\section{Introduzione}
L'obiettivo di questo progetto è calcolare l'energia dello stato fondamentale di un sistema di $^4\ce{He}$ in un potenziale esterno armonico con il metodo di Monte Carlo Variazionale. L'operatore hamiltoniano che descrive il sistema è dunque
\begin{equation*}
    H = -\frac{\hbar^2}{2m} \sum_{i = 1}^N \nabla_i^2 + \frac{1}{2} \omega^2 \sum_{i = 1}^N r_i^2 + \sum_{i < j} V(r_{ij}) 
    \quad\text{con}\quad 
    V(r) = 4\varepsilon\left[\left(\frac{\sigma}{r}\right)^{12} - \left(\frac{\sigma}{r}\right)^{6}\right],
\end{equation*}
che è il potenziale di \lj classico. Per usare unità di \unit{\angstrom} per le lunghezze e \unit{\kelvin} per le energie, nel caso dell'elio si hanno
\[
\varepsilon = 10.22\ \unit{\kelvin}, \quad
\sigma = 2.556\ \unit{\angstrom}, \quad 
\frac{\hbar^2}{2m} = 6.0596\ \unit{\square\angstrom\kelvin}.
\]
Sia $\Psi_{\boldsymbol{\alpha}}(\mathbf{R}) = \langle\mathbf{R}|\Psi_{\boldsymbol{\alpha}}\rangle$ una funzione d'onda parametrica per il sistema, in cui $\mathbf{R} = (\mathbf{r}_1, \mathbf{r}_2, \dots, \mathbf{r}_N)$ e $\boldsymbol{\alpha}$ è il set di parametri liberi. Il metodo variazionale permette di affermare che
\[
\min_{\boldsymbol{\alpha}} E_{\boldsymbol{\alpha}} = \min_{\boldsymbol{\alpha}} \frac{\langle\Psi_{\boldsymbol{\alpha}}|H|\Psi_{\boldsymbol{\alpha}}\rangle}{\langle\Psi_{\boldsymbol{\alpha}}|\Psi_{\boldsymbol{\alpha}}\rangle} \geq E_0
\]
ed $E_{\boldsymbol{\alpha}}$ si può valutare con una simulazione Monte Carlo usando $P(\mathbf{R}) = |\Psi_{\boldsymbol{\alpha}}(\mathbf{R})|^2$, mentre gli osservabili si calcoleranno con
\[
O_\boldalpha = \frac{1}{M}\sum_{k = 1}^M \frac{O\Psi_\boldalpha(\mathbf{R}_k)}{\Psi_\boldalpha(\mathbf{R}_k)}.
\]
La scelta intrapresa per la funzione d'onda, con $\boldalpha = (\alpha, \beta_1, \beta_2)$, è 
\[
\Psi_\boldalpha(\mathbf{R}) = \exp\left(-\frac{1}{2\alpha}\sum_{i = 1}^Nr_i^2-\frac{1}{2}\sum_{i < j}u_\beta(r_{ij})\right) \quad\text{con}\quad u_\beta(r) = \left(\frac{\beta_1}{r}\right)^{\beta_2}.
\]

\subsection{Energia cinetica}
Per la forma della funzione d'onda usata, conviene calcolare il contributo della particella i-esima all'energia cinetica nel seguente modo:
\[
T_i = -\frac{\hbar^2}{2m}\left(\nabla_i^2\log\Psi + (\boldsymbol{\nabla}_i\log\Psi)^2\right),
\]
che in funzione di $u_\beta(r)$ e delle sue derivate prima e seconda diventa
\[
T_i = \frac{\hbar^2}{2m}
\left[
\frac{3}{\alpha} + 
\frac{1}{2}\sum_{j\neq i} u_\beta''(r_{ij}) + 
\sum_{j\neq i} \frac{u_\beta'(r_{ij})}{r_{ij}} -
\frac{1}{\alpha^2}r_i^2 - 
\frac{1}{\alpha} \sum_{j\neq i} u_\beta'(r_{ij}) \mathbf{r}_i \cdot \hat{\mathbf{r}}_{ij} - 
\frac{1}{4}\left(\sum_{j\neq i} u_\beta'(r_{ij}) \hat{\mathbf{r}}_{ij}\right)^2
\right]
\]


\end{document}